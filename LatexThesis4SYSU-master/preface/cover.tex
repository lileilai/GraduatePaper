% !Mode:: "TeX:UTF-8"

%%  可通过增加或减少 setup/format.tex中的
%%  第274行 \setlength{\@title@width}{8cm}中 8cm 这个参数来 控制封面中下划线的长度。
\cheading{中山大学硕士学位论文}      % 设置正文的页眉,需要填上对应的毕业年份
\ctitle{基于消息传递机制的社会关系理解}    % 封面用论文标题,自己可手动断行
\etitle{Social Relationship Understanding via Message Passing Mechanism}    % 论文英文标题
\csubject{软件工程}   % 专业名称
\esubject{Computer Technology}
\cauthor{
李雷来
}            % 学生姓名
\eauthor{
Leilai Li
}
\csupervisor{
万海~副教授
}        % 导师姓名
\esupervisor{
Prof. Hai Wan
}

%\cdate{\the\year~\the\month~月~\the\day~日}
\cdate{二零一八~年~三~月~十四~日}

\cabstract{
本文致力于研究社会关系理解问题,社会关系理解是为了推断出一个给定场景中人之间的社会关系。近来关系理解在计算机视觉领域受到极大的关注,任务的效果也随着深度学习方法的发展得到了快速的提高,但是现有的工作主要通过挖掘人对的图像基本特征,或者引入物体和关系共现频率的先验知识来提升结果。这些工作将图片中的每个人对的关系检测独立的看待,并没有考虑到这些人对之间的相互关系。因此,在社会关系理解人物中很自然的考虑到这样的交互信息。例如,如果一张图片两个人对是朋友,那么第三个人对的关系往往是朋友或至少是其他亲密的社会关系,而不是无关系。因此,为了捕捉到这样交互的线索,本文提出一个端到端的可训练的人对关系网络,采用RNNs来实现人对之间的消息传递达到推理的目的,进而提高关系的分类结果。

在PPRN中,本文提出了一个消息传递和消息池化模块来实现不同人对之间的信息传递,达到不同人对间关系互相约束的目的,之后再实现了一个基于注意力机制来结合周边物体特征的模块。在这个过程中,首先进行的是人对关系之间的消息传递和池化这两个模块不断迭代,再融合区域生成网络生成的物体区域的图像特征,最后进行分类优化。

在实验中,本文在两个大规模数据集中验证了PPRN模型的有效性,这两个公开数据集包括三个不同的关系粒度,接着进行了具体案例的分析。实现结果表明了PPRN模型在与其他基准模型的对比中取得了最优的结果,同时说明了在视觉关系理解任务中考虑不同人对间相互影响的重要性。
}
\ckeywords{社会关系理解,消息传递,注意力机制,神经网络}

\eabstract{
This paper focuses on social relationships understanding which aims at inferring the social relations among people in a given scene. Relationship Understanding has attracted increasing attention in computer vision recently. Great progress has been made since the rise of deep learning. However, previous works mainly improves the results by mining the basic features of person pair or introducing prior knowledge of object and relationship co-occurrence frequencies, without taking into account the interaction of different pairs. It is natural to consider these interaction cues, {\it i.e.,} the mutual influence of multiple person pairs, in social relationship understanding. For instance, if two person pairs in an image are {\it Friends}, then the third pair is always {\it Friends} or at least other similar relations but not {\it No Relation}. Therefore, to capture these interaction cues, we propose a novel end-to-end trainable Person-Pair Relation Network (\sf{PPRN}) using standard RNNs, a inference network that learns iteratively to improve its predictions via message passing among person pair nodes.

In PPRN model, we provide a message passing and message pooling module to implements the message passing between various person pairs,achieving the purpose of the mutual restraint between different person pair, and we also implements a attention module to combine the contextual object feature. In this process, the first step is to iterate the two modules of message passing and pooling between person pair's relationships, and then combine the image features of object bounding box generated by the region proposal network, and finally optimization.

In the experiments, we evaluate our model in two large-scale datasets, and the two datasets contain three relational granularity, and further analyze by case studies. Experimental results demonstrate that our model outperforms baselines, which justifies the significance of considering the interaction between various person pairs in social relationship understanding.
}

\ekeywords{social relationship understanding,message passing,attention mechanism,neural network}

\makecover

\clearpage

