% !Mode:: "TeX:UTF-8"

\chapter{实验设计与分析}
\label{ch:exp}

本章将利用常见的社会关系理解的数据集对上一章提到的PPRN模型进行检测任务的验证,具体来说即对图片中两个标定的坐标人的关系类别。本章首先介绍当前所用到的两个数据集训练/验证/测试的数据分布情况,数据集的特点。再介绍若干对比模型,介绍实验的参数设置,然后分别对实验结果进行说明和分析,并且对其中消息池化进行了不同实现方法的对比。最后通过案例研究的方法来分析PPRN模型在关系补全任务中发挥的效果。

\section{数据集}

现有的大规模社会关系理解的数据集主要有两个:分别是PIPA-relation\cite{sun2017a}数据集和PISC\cite{li2017dual-glance}数据集,下面简单介绍这两个数据集。

PISC数据集全称是\textbf{People in social Context},它是Sun等人在2017年通过人工标注平台得到的数据集,这些图片主要来自Visual Genome\cite{krishna2017visual}、COCO\cite{lin2014microsoft}、YFCC100M\cite{thomee2016yfcc100m}、instagram和twitter等社交网站,Google和Bing商业搜索引擎。从这数据额的来源可以保证数据集的图片是足够高的方差的,人的面部表情,以及场景类型。PISC数据集包含22670张图片以及对应的社会关系标注,在PISC数据集上,又包含两个粒度的识别任务,coarse-level和fine-level。PIPA-relation数据集的全称是People in Photo Album Relation,总共包括37107张图片。
数据集的情况如表\ref{tab:exp-pisc-statistic},``Train''表示训练集图片的数量,``Valid''和``Test''分别表示验证和测试集的图片数量。``#train''表示训练集人对的数量,``#valid''和``#test''分别表示验证和测试集的人对数量。
\begin{table}[htpb]
  \centering
  \caption{PISC、PIPA-relation数据集的统计列表}
  \label{tab:exp-pisc-statistic}
  \begin{tabular}{c|c|c|c|c|c|c}
    \toprule
    数据集 & Train & Valid & Test & #train  &  #valid &  #test  \\
    \midrule
    PISC-coarse & 13142 & 4000 & 4000 & 14536 & 25636 & 15497   \\
    \midrule
    PISC-fine &  16828 & 500 & 1250 & 55400 & 1505 & 3691 \\
    \midrule
    PIPA-relation & 5857 & 261 & 2452 & 13729 & 709 & 5106 \\
    \bottomrule
  \end{tabular}
\end{table}