% !Mode:: "TeX:UTF-8"


\chapter{引言}
\label{ch:intro}

根据天津大学模板修改的符合中山大学毕业论文(至少是硕士论文)要求的Latex模板。

\section{使用方法}
\label{sec:usage}

本模板只包括内容方面的设计预定义,编译自行解决。作者使用的是Windows环境下MikTex+TeXstudio的组合。

\section{使用建议}
\label{sec:tips}

\subsection{普适问题}
\label{subsec:common}

普遍适用的论文排版问题:

\begin{itemize}
\item 图片标题在下,表格在上;一定要有标题,不能只是图1-1;与文字内容的间隔自行把握。
\item 参考文献建议使用.bib文件;也有使用Google Scholar的引用的,但有指出当中的“//”不符合规范。
\item 部分评审反馈,目录不包含摘要及目录本身,请根据情况自行斟酌。
\item 打印时需要右边翻页的问题(每章开始在右边页),可以在生成pdf后通过插入空白页解决(这样插入不会改变页码);或者尝试设置openright (未测试,有待探讨)。
\end{itemize}

\subsection{细节问题}
\label{subsec:specs}

一些细节的问题建议:
\begin{itemize}
\item 每个章节都有label,key使用ch:intro形式,以下使用sec:background等。图片key可以参考fig:scenes,表格参考tab:exp。
\item 图片、表格尽量在页的顶部,即float优先选择t。
\item 另外,为了打印时彩打方便,可以把需要彩打的图片尽量排版在一页,不过比较难调。
\item 虽然每个body的tex文件中包含了!Mode:: ``TeX:UTF-8"在文件开头,但仍有必要在IDE中将新建的tex文件设为UTF-8 编码,否则可能无法正常显示中文。
\end{itemize}

\subsection{其他说明}
\label{sec:setting}

参考文献\cite{wu2013online}目前采用上标表示。使用cite命令。

目前页眉设置:每章第一页页眉只有中间的“中山大学硕士毕业论文”,后续页左边显示“中山大学硕士毕业论文”,右边显示“第n章”。

目前页脚设置:仅包含页码,居中,无横线。

参考文献和附录计算页数,包含在目录,页眉设置同每章第一页。正文前的部分无页眉。

\section{例子}
\label{sec:examples}

图例子。label要在caption后。多图或子图方法上网查吧。

\begin{figure}[!t]
	\centering
	\includegraphics[width=0.95\textwidth]{scenes}
	\caption{图例}
	\label{fig:scenes}
\end{figure}

表例子。推荐使用这种三行表。缺省值使用三个“-”产生长横线“---”。

\begin{table}[!t]
\caption{示例表}
\label{tab:eg}
\vspace{0.5em}
\centering
\wuhao
	\begin{tabular}{ccccc}
	\toprule[1.5pt]
	表头 & 栏1 & 栏2 & 栏3 & 栏4 \\
	\midrule[1pt]
	内容1 & b & --- & $768 \times 576$ & 19 \\
	内容1 & a & 240/7 & $768 \times 576$ & --- \\
	\bottomrule[1.5pt]
	\end{tabular}
\end{table}

公式例子,与普通Latex数学公式无异。

\begin{equation}
1+1=2
\end{equation}

\subsection{研究背景和意义}

每个人的社会关系从构成了我们日常生活中社会结构的基础。自然的,我们利用一个人所在场景的社会关系来理解和解释当前的场景。社会学研究表明,这种对人的社会理解允许对其特征和可能的行为进行推断。当前,我们的社交生活很大部分是在社交媒体上,例如Facebook、Twitter、微信和微博等包含多模态信息的App,人们会通过文字、视频和音频等媒介含蓄的留下一些痕迹,但是我们能明确的扑捉到他们的社会关系通过分析多模态的信息。随着科技的发展和未来的到来,智能和潜在的自主系统会成为我们的帮手和同事,我们希望它们不仅可以熟练的完成任务,还希望他们能够融入和在我们人类生活的不同情况下采取适当的行动。此外,通过更好地了解这些隐藏信息,我们希望告知用户潜在的隐私风险。理解社会关系也有助于避免潜在的隐私风险,通过自动分析可能在文本等许多媒体中揭示社会关系的信息并告知用户这一点。在这个模式中,任务要求社会关系的概念和模式需要在生活和的所有方面共同努力,以便从一种感觉到的输入。虽然已经开始努力解决这一具有挑战性的问题,但社会生活的巨大多样性和复杂性阻碍了进展。最常见的,识别社会关系的计算模型仅仅限定于少数特定的类别。

在计算机视觉领域,社会关系信息被探索来提升几个常见的任务,例如人的轨迹预测\cite{kim2015brvo,robicquet2016learning}、 多目标追踪\cite{chen2012discovering,qin2012improving}和群体活动识别
\cite{direkoglu2012team,lan2012social,lan2012discriminative}。在图像理解任务上,视觉概念识别获得了越来越多的研究者的关注,包括视觉属性和视觉关系\cite{lu2016visual}。
视觉关系和视觉属性检测的主要目的是构建场景图谱,场景图谱(scene graph)\cite{johnson2015image}是对图片进行描述的一种半结构化的形式,场景图谱是由视觉三元组构成,并且包括关系三元组和属性三元组。场景图谱已经成为计算机视觉和人工智能领域的重要基础资源,因此如何自动的构建场景图谱成为了重要关注点,以利用自然语言信息的\cite{lu2016visual}为代表的工作,代表场景图谱自动生成领域取得了极大的进展。同样,社会属性和社会关系\cite{wang2010seeing} 对于场景理解同样重要。因此在当前工作,主要聚焦在解决社会关系检测问题上,并且可以从场景图谱的生成借鉴有用的思想。
给定一张图片,社会关系理解的目的是推断在当前图片这个场景下人之间的社会关系是社会关系检测的准确描述。除了前面提的用处,理解图像场景中这样的关系能帮助现有的算法产生更好的场景描述。例如在图\ref{fig:intro-example}中的第一个样例,用正常的文字来描述的话,`` 一个妇人和女孩正在吃饭''。但是对于社会关系的这个问题下,可以认为是``一个母亲和女儿正在吃饭''。
\begin{figure}[htpb]
	\centering
	%	\includegraphics[width=0.48 \textwidth, trim=10 10 10 80,clip]{./pic/example_new.pdf}
	\includegraphics[width=0.95 \textwidth,clip]{example-1.png}
	%\hspace{0.02\textwidth}
	%\vspace*{-0.08cm}
    \caption{PISC数据集中的一些图片例子}
	\vspace*{-3.5mm}
	\label{fig:intro-example}
\end{figure}
\looseness=-1

既然社会关系理解对于理提升上述任务的关键资源,那么自然而然的,如何准确的理解社会关系成为需要研究者需要攻克的课题。一方面,一张图片的社会关系可以通过众包的方式,人工标注得到,比如现有的数据集PISC\cite{li2017dual-glance}和PISC-relation\cite{sun2017a}。当然,自动端到端的方法包括基于人脸特征、年龄、人的头部特征等特征信息的\cite{sun2017a,zhang2015learning}。还有利用周边环境的信息的模型\cite{li2017dual-glance,wang2018deep},这些模型通常需要一个物体检测器或者检测器中RPN(region proposal network),这都是需要引入额外的标注框或者预训练模型。也有通过对周边物体和社会关系共现的统计,例如``computer''和``professional''共同出现的概率较大,如果识别出存在``computer'',那么当前的关系很大概率是``professional'',通过神经网络引入这些先验知识来提升预测的准确率。这些自动识别社会关系的模型虽然不断在进步,但是从实验结果来看,他们与人工标注的准确率还是存在很大鸿沟,离实际的应用还存在很大的距离。

然后,现有的学习模型大都倾向于利用外部的知识来辅助理解图片的社会关系场景,但是得到这些外部知识需要额外的人工标注,这是一件耗时耗力的工作,或者一些统计得到的先验知识同样包含一些噪音,这也直接引出了到底是否应该引入外部知识,例如是否利用周边物体的信息,以及如何在缺乏这些信息的情况下取得好的实验效果。受到场景图谱生成\cite{xu2017scene}的影响,xu(2017)的工作首先将整张图片输入,考虑到图片中不同视觉三元组之间的相互影响。例如,当知道``马在草地上''倾向于提高检测到``人骑着马''这条视觉三元组。对于社会关系检测的场景,如果图片中包含三个人对,其中两个人对的社会关系是``朋友'',那么第三条关系的的社会关系会倾向也是``朋友''或者其他的亲密关系,而不是``无关系''。直观上来说,这个是成立的,因此我们可以利用这当前场景下的其他的关系的来推理出当前的关系。


本论文主要研究如何将前文提到的关系场景的上下文信息引入社会关系理解的框架中。本论文完全区别于Li(2017)\cite{li2017dual-glance}和Wang(2018)\cite{wang2018deep}的工作,没有引入额外的检测标注
,但是基于Li等特征提取框架来提取特征。

