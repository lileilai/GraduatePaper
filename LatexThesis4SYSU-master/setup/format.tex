% !Mode:: "TeX:UTF-8"

%%%%%%%%%%%%%%%%% Fonts Definition and Basics %%%%%%%%%%%%%%%%%
\newcommand{\song}{\CJKfamily{song}}    % 宋体
\newcommand{\fs}{\CJKfamily{fs}}        % 仿宋体
\newcommand{\kai}{\CJKfamily{kai}}      % 楷体
\newcommand{\hei}{\CJKfamily{hei}}      % 黑体
\newcommand{\li}{\CJKfamily{li}}        % 隶书
\newcommand{\chuhao}{\fontsize{28pt}{28pt}\selectfont}       % 初号, 单倍行距
\newcommand{\yihao}{\fontsize{26pt}{26pt}\selectfont}       % 一号, 单倍行距
\newcommand{\xiaoyi}{\fontsize{24pt}{24pt}\selectfont}      % 小一, 单倍行距
\newcommand{\erhao}{\fontsize{22pt}{1.25\baselineskip}\selectfont}       % 二号, 1.25倍行距
\newcommand{\xiaoer}{\fontsize{18pt}{18pt}\selectfont}      % 小二, 单倍行距
\newcommand{\sanhao}{\fontsize{16pt}{16pt}\selectfont}      % 三号, 单倍行距
\newcommand{\xiaosan}{\fontsize{15pt}{15pt}\selectfont}     % 小三, 单倍行距
\newcommand{\sihao}{\fontsize{14pt}{14pt}\selectfont}       % 四号, 单倍行距
\newcommand{\xiaosi}{\fontsize{12pt}{12pt}\selectfont}      % 小四, 单倍行距
\newcommand{\wuhao}{\fontsize{10.5pt}{10.5pt}\selectfont}   % 五号, 单倍行距
\newcommand{\xiaowu}{\fontsize{9pt}{9pt}\selectfont}        % 小五, 单倍行距

\CJKtilde  % 重新定义了波浪符~的意义
\newcommand\prechaptername{第}
\newcommand\postchaptername{章}

\punctstyle{hangmobanjiao}             % 调整中文字符的表示,行内占一个字符宽度,行尾占半个字符宽度

% 调整罗列环境的布局
\setitemize{leftmargin=3em,itemsep=0em,partopsep=0em,parsep=0em,topsep=-0em}
\setenumerate{leftmargin=3em,itemsep=0em,partopsep=0em,parsep=0em,topsep=0em}

% 避免宏包 hyperref 和 arydshln 不兼容带来的目录链接失效的问题。
\def\temp{\relax}
\let\temp\addcontentsline
\gdef\addcontentsline{\phantomsection\temp}

% 自定义项目列表标签及格式 \begin{publist} 列表项 \end{publist}
\renewcommand{\labelenumi}{(\arabic{enumi})}               %定义列表项目符号为(1)格式
\newcounter{pubctr} %自定义新计数器
\newenvironment{publist}{%%%%%定义新环境
\begin{list}{[\arabic{pubctr}]} %%标签格式
    {
     \usecounter{pubctr}
     \setlength{\leftmargin}{2.5em}   % 左边界 \leftmargin =\itemindent + \labelwidth + \labelsep
     \setlength{\itemindent}{0em}     % 标号缩进量
     \setlength{\labelsep}{1em}       % 标号和列表项之间的距离,默认0.5em
     \setlength{\rightmargin}{0em}    % 右边界
     \setlength{\topsep}{0ex}         % 列表到上下文的垂直距离
     \setlength{\parsep}{0ex}         % 段落间距
     \setlength{\itemsep}{0ex}        % 标签间距
     \setlength{\listparindent}{0pt}  % 段落缩进量
    }}
{\end{list}}

\makeatletter
\renewcommand\normalsize{
  \@setfontsize\normalsize{12pt}{12pt} % 小四对应 12 pt
  \setlength\abovedisplayskip{4pt}
  \setlength\abovedisplayshortskip{4pt}
  \setlength\belowdisplayskip{\abovedisplayskip}
  \setlength\belowdisplayshortskip{\abovedisplayshortskip}
  \let\@listi\@listI}
%\def\defaultfont{\renewcommand{\baselinestretch}{2.0}\normalsize\selectfont} % 设置行距
\def\defaultfont{\renewcommand{\baselinestretch}{2.0}\normalsize}
\renewcommand{\CJKglue}{\hskip -0.1 pt plus 0.08\baselineskip} % 控制字间距,使每行 34 个汉字
\makeatother
\setlength{\parskip}{3pt plus.1pt minus.1pt}  % 段落之间的竖直距离


%%%%%%%%%%%%% Contents %%%%%%%%%%%%%%%%%
\renewcommand{\contentsname}{目\qquad 录}
\setcounter{tocdepth}{1} % 控制目录深度   //只显示两级目录
\titlecontents{chapter}[2em]{\vspace{.5\baselineskip}\sihao\song}
             %{\prechaptername\CJKnumber{\thecontentslabel}\postchaptername\qquad}{}   %让目录章节使用“第一章”
             {\prechaptername~\thecontentslabel~\postchaptername\quad}{}               %让目录章节使用“第1章”
             {\hspace{.5em}\titlerule*[5pt]{$\cdot$}\xiaosi\contentspage}
\titlecontents{section}[3em]{\vspace{.25\baselineskip}\xiaosi\song}
             {\thecontentslabel\quad}{}
             {\hspace{.5em}\titlerule*[5pt]{$\cdot$}\xiaosi\contentspage}
\titlecontents{subsection}[4em]{\vspace{.25\baselineskip}\xiaosi\song}
             {\thecontentslabel\quad}{}
             {\hspace{.5em}\titlerule*[5pt]{$\cdot$}\xiaosi\contentspage}

%%%%%%%%%% Chapter and Section %%%%%%%%%%%%%
\setcounter{secnumdepth}{4}
\setlength{\parindent}{2em}
\renewcommand{\chaptername}{\prechaptername\CJKnumber{\thechapter}\postchaptername}     %章名使用“第一章”
\renewcommand{\chaptername}{\prechaptername~\thechapter~\postchaptername}               %章名使用“第1章”
%\titleformat{\chapter}{\centering\xiaoer\bfseries}{\hei\chaptername}{2em}{}
\titleformat{\chapter}{\centering\erhao\hei}{\chaptername}{2em}{}
\titlespacing{\chapter}{0pt}{0.1\baselineskip}{0.8\baselineskip}
\titleformat{\section}{\xiaosan\hei}{\thesection}{1em}{}
\titlespacing{\section}{0pt}{0.15\baselineskip}{0.25\baselineskip}
\titleformat{\subsection}{\sihao\hei}{\thesubsection}{1em}{}
\titlespacing{\subsection}{0pt}{0.1\baselineskip}{0.3\baselineskip}
\titleformat{\subsubsection}{\sihao\hei}{\thesubsubsection}{1em}{}
\titlespacing{\subsubsection}{0pt}{0.05\baselineskip}{0.1\baselineskip}


%%%%%%%%%% Table, Figure and Equation %%%%%%%%%%%%%%%%%
\renewcommand{\tablename}{表}                                     % 插表题头
\renewcommand{\figurename}{图}                                    % 插图题头
\renewcommand{\thefigure}{\arabic{chapter}-\arabic{figure}}       % 使图编号为 7-1 的格式 %\protect{~}
%\renewcommand{\thesubfigure}{\alph{subfigure})}                  % 使子图编号为 a) 的格式
\renewcommand{\thesubfigure}{(\alph{subfigure})}                  % 使子图编号为 (a) 的格式
\renewcommand{\thesubtable}{(\alph{subtable})}                    % 使子表编号为 (a) 的格式
\renewcommand{\thetable}{\arabic{chapter}-\arabic{table}}         % 使表编号为 7-1 的格式
\renewcommand{\theequation}{\arabic{chapter}-\arabic{equation}}   % 使公式编号为 7-1 的格式

\makeatletter
	% 使子图引用也是7-1a)或7-1(a)的形式
	\renewcommand{\p@subfigure}{\thefigure}
\makeatother

%%%%%% 定制浮动图形和表格标题样式 %%%%%%
\makeatletter
\long\def\@makecaption#1#2{
   \vskip\abovecaptionskip
   \sbox\@tempboxa{\centering\wuhao\song{#1\quad #2} }
   \ifdim \wd\@tempboxa >\hsize
     \centering\wuhao\song{#1\qquad #2} \par
   \else
     \global \@minipagefalse
     \hb@xt@\hsize{\hfil\box\@tempboxa\hfil}
   \fi
   \vskip\belowcaptionskip}
\makeatother
\captiondelim{~~~~} %用来控制longtable表头分隔符

\setlength{\floatsep}{10pt plus 3pt minus 2pt}      % 图形之间或图形与正文之间的距离
\setlength{\abovecaptionskip}{5pt plus.1pt minus.1pt} % 图形中的图与标题之间的距离
\setlength{\belowcaptionskip}{3pt plus.1pt minus.1pt} % 表格中的表与标题之间的距离


%%%%%%%%%% Theorem Environment %%%%%%%%%%%%%%%%%
\theoremstyle{plain}
\theorembodyfont{\song}%\rmfamily}
\theoremheaderfont{\hei\bfseries}
\newtheorem{theorem}{定理~}[chapter]
\newtheorem{lemma}{引理~}[chapter]
\newtheorem{axiom}{公理~}[chapter]
\newtheorem{proposition}{命题~}[chapter]
\newtheorem{prop}{性质~}[chapter]
\newtheorem{corollary}{推论~}[chapter]
\newtheorem{conclusion}{结论~}[chapter]
\newtheorem{definition}{定义~}[chapter]
\newtheorem{conjecture}{猜想~}[chapter]
\newtheorem{example}{例~}[chapter]
\newtheorem{remark}{注~}[chapter]
%\newtheorem{algorithm}{算法~}[chapter]
\newenvironment{proof}{\noindent{\hei 证明:}}{\hfill $ \square $ \vskip 4mm}
\theoremsymbol{$\square$}

\renewcommand{\algorithmcfname}{算法}


%%%%%%%%%% Page: number, header and footer  %%%%%%%%%%%%%%%%%

%\frontmatter 或 \pagenumbering{roman}
%\mainmatter 或 \pagenumbering{arabic}
\makeatletter
\renewcommand\frontmatter{\clearpage
  \@mainmatterfalse
  }
\makeatother

%%%%%%%%%%%% References %%%%%%%%%%%%%%%%%
\renewcommand{\bibname}{参考文献}
% 重定义参考文献样式,来自thu
\makeatletter
\renewenvironment{thebibliography}[1]{
    \titleformat{\chapter}{\raggedright\sihao\hei}{\chaptername}{2em}{}
   \chapter*{\bibname}
   \wuhao
   \list{\@biblabel{\@arabic\c@enumiv}}
        {\renewcommand{\makelabel}[1]{##1\hfill}
         \settowidth\labelwidth{0 cm}
         \setlength{\labelsep}{0pt}
         \setlength{\itemindent}{0pt}
         \setlength{\leftmargin}{\labelwidth+\labelsep}
         \addtolength{\itemsep}{-0.7em}
         \usecounter{enumiv}
         \let\p@enumiv\@empty
         \renewcommand\theenumiv{\@arabic\c@enumiv}}
    \sloppy\frenchspacing
    \clubpenalty4000
    \@clubpenalty \clubpenalty
    \widowpenalty4000
    \interlinepenalty4000
    \sfcode`\.\@m}
   {\def\@noitemerr
     {\@latex@warning{Empty `thebibliography' environment}}
    \endlist\frenchspacing}
\makeatother

\addtolength{\bibsep}{-0.5em}     % 缩小参考文献间的垂直间距
\setlength{\bibhang}{2em}         % 每个条目自第二行起缩进的距离



% 参考文献引用作为上标出现
%\newcommand{\citeup}[1]{\textsuperscript{\cite{#1}}}
\makeatletter
    \def\@cite#1#2{\textsuperscript{[{#1\if@tempswa , #2\fi}]}}
\makeatother
%% 引用格式
\bibpunct{[}{]}{,}{s}{}{,}

%%%%%%%%%%%% Cover %%%%%%%%%%%%%%%%%
% 封面、摘要、版权、致谢格式定义
\makeatletter
\def\ctitle#1{\def\@ctitle{#1}}\def\@ctitle{}
\def\etitle#1{\def\@etitle{#1}}\def\@etitle{}
\def\csubject#1{\def\@csubject{#1}}\def\@csubject{}
\def\esubject#1{\def\@esubject{#1}}\def\@esubject{}
\def\cauthor#1{\def\@cauthor{#1}}\def\@cauthor{}
\def\eauthor#1{\def\@eauthor{#1}}\def\@eauthor{}
\def\csupervisor#1{\def\@csupervisor{#1}}\def\@csupervisor{}
\def\esupervisor#1{\def\@esupervisor{#1}}\def\@esupervisor{}
\def\cdate#1{\def\@cdate{#1}}\def\@cdate{}
\long\def\cabstract#1{\long\def\@cabstract{#1}}\long\def\@cabstract{}
\long\def\eabstract#1{\long\def\@eabstract{#1}}\long\def\@eabstract{}
\def\ckeywords#1{\def\@ckeywords{#1}}\def\@ckeywords{}
\def\ekeywords#1{\def\@ekeywords{#1}}\def\@ekeywords{}
\def\cheading#1{\def\@cheading{#1}}\def\@cheading{}


\pagestyle{fancy}
  \fancyhf{}
  \renewcommand{\chaptermark}[1]%
    {\markboth{\chaptername \ #1}{}}            % \chaptermark 去掉章节标题中的数字
  %\fancyhead[C]{\song\wuhao \@cheading}  % 页眉
  %\fancyhead[RO]{\song\xiaosi \leftmark}
  %\fancyhead[LE]{\song\xiaosi \@ctitle}
  \lhead{\song\wuhao \@ctitle}  % 左页眉
  \rhead{\song\wuhao \leftmark}    % 右页眉
  \fancyfoot[C]{\song\wuhao ~\thepage~}
\newlength{\@title@width}

% 定义封面
\def\makecover{
%\cleardoublepage%
   \phantomsection
    \pdfbookmark[-1]{\@ctitle}{ctitle}

\begin{titlepage}
\vspace*{10pt}
\begin{center}

  \vspace*{10pt}
  \hei\chuhao{\textbf{中山大学硕士学位论文}}

  \vspace*{60pt}
  \song\xiaoer\textbf{\@ctitle}

  \xiaoer{\textbf{\@etitle}}

  \begin{spacing}{2.0}
  \vspace*{42pt}
  \setlength{\@title@width}{6cm}
  {\sihao\kai{{

  \begin{tabular}{lc}
    学~~位~~申~~请~~人:   &  \underline{\makebox[\@title@width][c]{\@cauthor}} \\
    导师姓名及职称:       &  \underline{\makebox[\@title@width][c]{\@csupervisor}} \\
    专~~~~业~~~~名~~~~称: &  \underline{\makebox[\@title@width][c]{\@csubject}}\\
  \end{tabular}}}
 }

 \end{spacing}

  \vspace*{10pt}


 \begin{spacing}{2.0}
 \vspace*{10pt}
  \setlength{\@title@width}{5cm}
  {\sanhao\kai{{
  \begin{tabular}{lc}
    答辩委员会主席(签名):  &  \underline{\makebox[\@title@width][c]{~}} \\
    答辩委员会委员(签名):  &  \underline{\makebox[\@title@width][c]{~}} \\
    ~ &  \underline{\makebox[\@title@width][c]{~}}\\
    ~ &  \underline{\makebox[\@title@width][c]{~}}\\
    ~ &  \underline{\makebox[\@title@width][c]{~}}\\
    ~ &  \underline{\makebox[\@title@width][c]{~}}\\
  \end{tabular}}}
 }
 \end{spacing}
 \vspace*{20pt}

  \vspace*{21pt}

\kai\sihao{\@cdate}
\end{center}
\end{titlepage}

 %空白页
\newpage
\thispagestyle{empty}
\mbox{}

%%%%%%%%%%%%%%%%   Originality Statement  %%%%%%%%%%%%%%%%%%%%%%%
\clearpage
\pdfbookmark[0]{论文原创性声明}{originality}
\chapter*{\centering\sanhao\song\bfseries 论文原创性声明}
\song\defaultfont
本人郑重声明:所呈交的学位论文,是本人在导师的指导下,独立进行研究工作所取得的成果。除文中已经注明引用的内容外,本论文不包含任何其他个人或集体已经发表或撰写过的作品成果。对本文的研究作出重要贡献的个人和集体,均已在文中以明确方式标明。本人完全意识到本声明的法律结果由本人承担。

\vspace*{40pt}
\begin{flushright}
\setlength{\@title@width}{5cm}
  {\sihao\song{
  \begin{tabular}{lc}
    学位论文作者签名:           &  \underline{\makebox[\@title@width][c]{~}} \\
    \qquad\qquad\qquad 日~~期:  &  \underline{\makebox[\@title@width][c]{~}} \\
  \end{tabular}}
 }
\end{flushright}

%%%%%%%%%%%%%%%%   Authorization Statement  %%%%%%%%%%%%%%%%%%%%%%%
\vspace*{60pt}
\pdfbookmark[0]{学位论文使用授权声明}{authorization}
\begin{center}
  \sanhao\song\bfseries{学位论文使用授权声明}
\end{center}

\song\defaultfont
本人完全了解中山大学有关保留、使用学位论文的规定,即:学校有权保留学位论文并向国家主管部门或其指定机构送交论文的电子版和纸质版,有权将学位论文用于非赢利目的的少量复制并允许论文进入学校图书馆、院系资料室被查阅,有权将学位论文的内容编入有关数据库进行检索,可以采用复印、缩印或其他方法保存学位论文。

\vspace*{40pt}
\begin{center}
\setlength{\@title@width}{5cm}
  {\sihao\song{
  \begin{tabular}{ll}
    学位论文作者签名: \qquad\qquad\qquad\qquad\qquad  &  导师签名: \qquad\qquad\qquad\\
    日期: \qquad 年\qquad 月\qquad 日     &  日期: \qquad 年\qquad 月\qquad 日 \\
  \end{tabular}}
 }
\end{center}
\thispagestyle{empty}   % 去掉页码

 %空白页
\newpage
\thispagestyle{empty}
\mbox{}


%%%%%%%%%%%%%%%%%%%   Abstract and Keywords  %%%%%%%%%%%%%%%%%%%%%%%
\clearpage
\setcounter{page}{1} % 重新开始页码
\pagenumbering{Roman} %罗马数字页码
\markboth{摘~要}{摘~要}
\pdfbookmark[0]{摘~~要}{cabstract}
\newpage

\begin{flushleft}
\setlength{\@title@width}{5cm}
  {\wuhao\song{
  \begin{tabular}{ll}
    论文题目:  & \@ctitle  \\ %&  %\@ctitle\\
    专~~~~~~业:  &  \@csubject \\
    硕~~士~~生:  &  \@cauthor \\
    指导老师:    &  \@csupervisor \\
  \end{tabular}}
 }
\end{flushleft}

%\addcontentsline{toc}{chapter}{摘~要}
\vspace{\baselineskip}
\begin{center}
\xiaoer\hei\bfseries 摘\qquad 要
\end{center}

\song\defaultfont
\@cabstract
\vspace{\baselineskip}

\hangafter=1\hangindent=52.3pt\noindent
{\hei\xiaosi 关键词:} \@ckeywords
%\thispagestyle{plain}

% 空白页
\clearpage{\pagestyle{empty}\cleardoublepage}
%\newpage
%\thispagestyle{empty}
%\mbox{}

%%%%%%%%%%%%%%%%%%%   English Abstract  %%%%%%%%%%%%%%%%%%%%%%%%%%%%%%
\clearpage
%\fancypagestyle{plain}{                             %设置英文摘要页眉的论文题目为英文
%\fancyhead{}
%\fancyhead[RO]{\wuhao \@etitle}  % 左页眉
%\fancyhead[LE]{\wuhao \leftmark}    % 右页眉
%\fancyfoot[C]{\song\xiaowu~\thepage~}
%}
%\thispagestyle{plain}
%\phantomsection
\markboth{ABSTRACT}{ABSTRACT}
\pdfbookmark[0]{ABSTRACT}{eabstract}
%\addcontentsline{toc}{chapter}{ABSTRACT}   % 摘要不加到目录中
\newpage

\begin{flushleft}           % 如果英文title太长,手动分成两行
\setlength{\@title@width}{5cm}
  {\wuhao{
  \begin{tabular}{lp{0.8\textwidth}}
    Title:  & \@etitle \\
    Major:      &  \@esubject \\
    Name:       &  \@eauthor \\
    Supervisor: &  \@esupervisor \\
  \end{tabular}}
 }
\end{flushleft}

\vspace{\baselineskip}
\begin{center}
\sanhao{\bf{Abstract}}
\end{center}

%\vspace{\baselineskip}
\@eabstract
\vspace{\baselineskip}

\hangafter=1\hangindent=52.3pt\noindent
{\textbf{Keywords:}} \@ekeywords
%\thispagestyle{plain}

% 空白页
\clearpage{\pagestyle{empty}\cleardoublepage}
}
\makeatother
